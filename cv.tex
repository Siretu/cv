%%%%%%%%%%%%%%%%%%%%%%%%%%%%%%%%%%%%%%%%%
% Plasmati Graduate CV
% LaTeX Template
% Version 1.0 (24/3/13)
%
% This template has been downloaded from:
% http://www.LaTeXTemplates.com
%
% Original author:
% Alessandro Plasmati (alessandro.plasmati@gmail.com)
%
% Edited by:
% Erik Ihrén (erikihr@gmail.com)
%
% License:
% CC BY-NC-SA 3.0 (http://creativecommons.org/licenses/by-nc-sa/3.0/)
%
% Important note:
% This template needs to be compiled with XeLaTeX.
% The main document font is called Fontin and can be downloaded for free
% from here: http://www.exljbris.com/fontin.html
%
%%%%%%%%%%%%%%%%%%%%%%%%%%%%%%%%%%%%%%%%%

%----------------------------------------------------------------------------------------
%	PACKAGES AND OTHER DOCUMENT CONFIGURATIONS
%----------------------------------------------------------------------------------------

\documentclass[a4paper,10pt]{article} % Default font size and paper size

\usepackage{enumitem}
\usepackage{xstring}

\usepackage{fancyhdr}
\rhead{}
\renewcommand{\headrulewidth}{0pt}
\renewcommand{\footrulewidth}{0pt}


\usepackage{fontspec} % For loading fonts
\defaultfontfeatures{Mapping=tex-text}
\setmainfont[SmallCapsFont = Fontin SmallCaps]{Fontin} % Main document font

\usepackage{xunicode,xltxtra,url,parskip} % Formatting packages

\usepackage[usenames,dvipsnames]{xcolor} % Required for specifying custom colors

\usepackage{fullpage} % Margin formatting of the A4 page, an alternative to layaureo can be \usepackage{fullpage}
% To reduce the height of the top margin uncomment: 
%\addtolength{\voffset}{-1.3cm}
%\setlength{textheight}{700pt}

\usepackage{hyperref} % Required for adding links	and customizing them
\definecolor{linkcolour}{rgb}{0,0.2,0.6} % Link color
\hypersetup{colorlinks,breaklinks,urlcolor=linkcolour,linkcolor=linkcolour} % Set link colors throughout the document

\usepackage{titlesec} % Used to customize the \section command
\titleformat{\section}{\Large\scshape\raggedright}{}{0em}{}[\titlerule] % Text formatting of sections
\titlespacing{\section}{0pt}{1.5pt}{1.5pt} % Spacing around sections

\newcommand{\anonymous}{false}
\newcommand{\longdesc}{false}

\usepackage[
top    = 0.5cm,
bottom = 0.4cm,
left   = 2.50cm,
right  = 2.50cm
]{geometry}
%\newgeometry{left=3cm,bottom=0.1cm}

\begin{document}

\pagestyle{fancy} % Removes page numbering
\pagenumbering{gobble}

%\font\fb=''[cmr10]'' % Change the font of the \LaTeX command under the skills section

%----------------------------------------------------------------------------------------
%	NAME AND CONTACT INFORMATION
%----------------------------------------------------------------------------------------


\IfEqCase{\anonymous}{
	{false}{\par{\centering{\huge Erik Ihrén\\\bigskip
\small \textbf{Location:} Stockholm, Sweden. \textbf{GitHub:} \href{https://github.com/Siretu}{Siretu} \\ \textbf{Email:} erikihr@gmail.com \textbf{Phone:} (+46) 72 213 1206
}\par}}
	{true}{\par{\centering{\huge John Doe\\\bigskip
\small \textbf{Location:} Somewhere, Sweden. \textbf{GitHub:} \href{https://github.com/Someone}{Someone} \\ \textbf{Email:} something@gmail.com \textbf{Phone:} (+46) 70 123 4567
}\par}}
}


%----------------------------------------------------------------------------------------
%	EDUCATION
%----------------------------------------------------------------------------------------

\section{Education}
\begin{tabular}{rl}	
\textsc{2012 - } & \textbf{\IfEqCase{\anonymous}{
	{false}{Royal Institute of Technology, Stockholm}
	{true}{Some University, Sweden}
}}\\
& Master's Degree in \textsc{Computer Science}\\
& Expected graduation year: 2017\\
& Major GPA: 3.92\\

%\textsc{2009 - 2012} & High School at \textbf{Danderyds Gymnasium}\\&Advanced placement program for Mathematics\\
\end{tabular}
%\footnotetext[1]{Converted from ECTS using \url{http://www.foreigncredits.com/Resources/GPA-Calculator/Sweden}}

%----------------------------------------------------------------------------------------
%	WORK EXPERIENCE 
%----------------------------------------------------------------------------------------

\section{Work Experience}
\begin{tabular}{lp{13cm}}
\textbf{Ericsson} & Software Developer, June 2014 - December 2014 \\
& \IfEqCase{\longdesc}{
	{true}{\small{Worked on automating dependency extraction for Ericsson’s entire RNC codebase. The work I did automatically figures out the impacted projects and automatically rebuilds and tests those when a commit is pushed. Also improved upon a general build support tool to simplify continuous integration.}}
	{false}{\small{Worked on automating dependency extraction for Ericsson’s entire RNC codebase. The work I did automatically figures out the impacted projects and automatically rebuilds and tests those when a commit is pushed. Also improved upon a general build support tool to simplify continuous integration.}}
}
\\
\multicolumn{2}{c}{} \\

%------------------------------------------------

\textbf{Domain Name ~~~~~~} & Summer Intern, May 2013 - August 2013\\
\textbf{Services     }& \IfEqCase{\longdesc}{
	{true}{\small{Worked in South Africa on programming an internal asset registry in the Django admin interface for registering and keeping track of company assets. I also participated in the ICANN and AFRINIC conferences where I spoke to potential customers about dotAfrica.}}
	{false}{\small{Worked in South Africa on programming an internal asset registry in the Django admin interface for registering and keeping track of company assets. I also participated in the ICANN and AFRINIC conferences where I spoke to potential customers about dotAfrica.}}
}\\
\multicolumn{2}{c}{} \\

%------------------------------------------------

\textbf{Netnod} & Summer Intern, June 2011 - August 2011 and June 2012 - August 2012\\
 & \IfEqCase{\longdesc}{
 	{true}{\small{Developed a user interface with Python’s framework Django to help visualize arbitrary data from databases. This was used to automatically translate and present data for Netnod customers in the Django Admin Interface. I came back a second year after that to optimize and maintain my work.}}
 	{false}{\small{Developed a user interface with Python’s framework Django to help visualize arbitrary data from databases. This was used to automatically translate and present data for Netnod customers in the Django Admin Interface.}}
 }\\
 \multicolumn{2}{c}{} \\
 
 %------------------------------------------------
 
 \textbf{KTH} & Teaching Assistant, August 2014 - December 2014\\
 & \small{Worked as TA for the course Programming Paradigms where students learn about functional and logic programming in Haskell and Prolog. }  
 
\end{tabular}


%----------------------------------------------------------------------------------------
%	PROJECTS
%----------------------------------------------------------------------------------------

\section{Projects}
\begin{tabular}{lp{13cm}}
& 
\IfEqCase{\anonymous}{
	{false}{\footnotesize{\emph{Full portfolio can be found on \href{http://www.nada.kth.se/~eihren}{my personal site.}}}}
	{true}{\footnotesize{\emph{Full portfolio can be found on \href{http://www.somesite.com}{my personal site.}}}}
}
 \\\\
\textbf{\href{https://betafamily.com/webrecorder/demo}{SuperWebRecorder}} & \IfEqCase{\longdesc}{
	{true}{\small{Worked as a Project Manager and Programmer in a project with 8 other people for The Beta Family to develop their WebRecorder product that records users on websites for beta testing and usability studies. I focused on serializing all user interaction, along with site DOM changes that could then be saved in a database and replayed as a recording.}}
	{false}{\small{Developed a project along with 8 other people for The Beta Family.
	\begin{itemize}
		\item Worked as the Project Manager, leading 8 other people while simultaneously participating heavily in programming.
		\item Serialized webpage DOM state as JSON object and sent to database to be replayed later.
	\end{itemize}
	}}
}
	  \\
\IfEqCase{\anonymous}{
	{false}{\textbf{\href{http://www.nada.kth.se/~eihren/\#cruisercommand}{Cruiser Command}}}
	{true}{\textbf{\href{http://www.cruisercommand.com}{Cruiser Command}}}
} &\IfEqCase{\longdesc}{
	{true}{ \small{Spent three years continuously developing and maintaining a mod on my own for Starcraft 2 played by thousands of players with an average rating of over 4.8/5 out of over 2000 reviews. While developing Cruiser Command from start to finish, I had to solve interesting problems such as advanced Artificial Intelligence, complex math intersection formulas and tracking down bugs in a large code base.}}
	{false}{\small{Spent three years continuously developing my own mod for Starcraft 2, played by thousands.
	\begin{itemize}
		\item Average rating of over 4.8/5 out of over 2000 reviews.
		\item Implementing advanced AI for a complex team-based strategy game. 
		\item Working and fixing bugs in a code base with over 25,000 lines of my own code.
	\end{itemize}
	}}
} \\
%\textbf{Quadcopter:} & \small{Built and programmed a Quadrocopter from scratch during the summer of 2010 with a friend using the open source AeroQuad library.} \\\\
\IfEqCase{\anonymous}{
	{false}{\textbf{\href{https://github.com/Siretu/kartobot}{Kartobot}}}
	{true}{\textbf{\href{http://github.com/Someone/source}{Kartobot}}}
} & \IfEqCase{\longdesc}{
	{true}{\small{Built and programmed a mapping robot with two friends. Using a tower with rotating ultrasonic sensors it was possible to measure the distance and use that to draw a 2D map of the room, while it was exploring using a pathfinding algorithm. I worked mainly on the tower and making sure the ultrasonic sensors were working properly. I also developed an A* algorithm in C for the pathfinding and ended up optimizing it by 100x.}}
	{false}{\small{Built and programmed a mapping robot with two friends. Uses ultrasonic sensors to draw a map.
	\begin{itemize}
		\item Implemented A* pathfinding algorithm in C
		\item Optimized algorithm for extreme memory constraints (10kB)
		\item Modified algorithm to use binary heaps to increase performance with 100x.
	\end{itemize}
	}}}
\end{tabular}


\titlespacing{\section}{0pt}{-12pt}{0pt} % Pull Computer skills closer

%----------------------------------------------------------------------------------------
%	COMPUTER SKILLS 
%----------------------------------------------------------------------------------------

\section{Computer Skills}
\begin{tabular}{lp{13cm}}
\textbf{Languages} & \small{I'm experienced with Python, Java, C\# and C/C++. I'm also familiar with a great deal of other languages like Lisp, Go, Haskell, Prolog and Javascript.}\\\\
\textbf{Tools/Frameworks~} & \small{I have experience with following tools and frameworks: Git, CAD, Django, SQL, HTML/CSS and LaTeX.}
%\\\\
%\textsc{Python:} & \small{Used from an early age as my go-to language. I’ve created a great number of pet projects in Python and it was also the language I used the most during my previous internships.}\\\\

%\textsc{Java:} & \small{The second language I learned. I’ve used it for over four years and used it to develop some larger applications (compared to my projects in Python).}\\\\

%\textsc{Miscellaneous:} & \small{I also have experience with the following languages/frameworks/tools:  C/C++, C\#, Lisp, Go, Haskell, Prolog, Javascript, Git, CAD, Django, SQL and LaTeX}
\end{tabular}



%----------------------------------------------------------------------------------------
%	PRIZES & AWARDS
%----------------------------------------------------------------------------------------

\section{Prizes and awards}
\begin{itemize}[noitemsep,topsep=0pt,parsep=0pt,partopsep=0pt]
\small{
\item Finalist in Swedish national programming competition at high school level.
\item First place in Linköping University’s case competition 2011.
\item Second place in Spotify’s Hockey Programming Challenge.
}
\end{itemize}
\end{document}
